\documentclass[a4paper,11pt]{report}
\usepackage{amssymb}
%%%%%%%%%%%%%%%%%%%%%%%%%%%%%%%%%%%%%%%%%%%%%%%%%%%%%%%%%%%%%%%%%%%%%%%%%%%%%%%%%%%%%%%%%%%%%%%%%%%%%%
% !TEX encoding = UTF-8 Unicode
\usepackage{UFMG_style}
%\usepackage{oxthesis}
\usepackage{etexcmds,epsfig}
\usepackage{amssymb, amsmath, multicol, rotating, xcolor} 
\usepackage{graphicx, balance}
\usepackage{subfigure}% Support for small, `sub' figures and tables
\usepackage{ccaption}
\usepackage{amsfonts}  %% 100
\usepackage{graphics}  %% 100
\DeclareGraphicsExtensions{.pdf,.svg,.eps,.png,.jpg,.jpeg}
\usepackage[portuges]{varioref}
\usepackage[utf8]{inputenc}
\usepackage{url} 
\usepackage[T1]{fontenc}
\usepackage{dirtytalk} % quotations
\usepackage{makeidx}
\usepackage[portuges,brazilian]{babel}
\usepackage{hyphenat}
\usepackage{tabularx,balance,indentfirst}
\setcounter{MaxMatrixCols}{10}
\usepackage{floatrow}
\usepackage{chngcntr}
\usepackage{microtype}
%\usepackage[xindy,toc,acronym]{glossaries}
%%%%%%%%
\makeatletter
\let\X@old@caption\caption
\def\X@caption@minusone{\expandafter\advance\csname c@\@captype\endcsname-1 }
\def\X@caption@br[#1]#2{\X@old@caption[#1]{#2}\X@caption@minusone}
\def\X@caption@nobr#1{\X@old@caption{#1}\X@caption@minusone}
\def\caption{\@ifnextchar[\X@caption@br\X@caption@nobr}
\makeatother
%%%%%%%%
% substitui Verbatim
\usepackage{listings}
\lstset{numbers=left,commentstyle=\color{green},keywordstyle=\color{blue}}
\usepackage{inconsolata}
\lstset{
    language=matlab, %% Troque para PHP, C, Java, etc... bash é o padrão
    basicstyle=\ttfamily\small,
    numberstyle=\footnotesize,
    numbers=left,
    %backgroundcolor=\color{gray!1}, %estava 10
    frame=single,
    tabsize=2,
    rulecolor=\color{black!30},
    title=\lstname,
    escapeinside={\%*}{*)},
    breaklines=true,
    breakatwhitespace=true,
    framextopmargin=2pt,
    framexbottommargin=2pt,
    extendedchars=false,
    showstringspaces=false,
    inputpath=./programas/,
    %inputencoding=utf8
}
%------------------------------------------------------%
\newtheorem{teorema}{Teorema}
\newtheorem{algoritmo}{Algoritmo}
\newtheorem{axioma}{Axioma}
\newtheorem{caso}{Caso}
\newtheorem{conclusão}{Conclusão}
\newtheorem{condição}{Condição}
\newtheorem{conjectura}{Conjectura}
\newtheorem{corol\'{a}rio}{Corol\'{a}rio}
\newtheorem{crit \'{e}rio}{Crit \'{e}rio}
\newtheorem{definicao}{Definicao}
\newtheorem{definição}{Definição}
\newtheorem{exemplo}{Exemplo}
\newtheorem{exerc\'{\i}cio}{Exerc\'{\i}cio}
\newtheorem{lema}{Lema}
\newtheorem{nota\c{c}\~{a}o}{Nota\c{c}\~{a}o}
\newtheorem{observa\c{c}\~{a}o}{Observa\c{c}\~{a}o}
\newtheorem{problema}{Problema}
\newtheorem{proposi\c{c}\~{a}o}{Proposi\c{c}\~{a}o}
\newtheorem{solu\c{c}\~{a}o}{Solu\c{c}\~{a}o}
\newtheorem{resumo}{Resumo}
\newenvironment{prova}[1][Prova]{\textbf{#1.} }{\ \rule{0.5em}{0.5em}}
\renewcommand{\baselinestretch}{1.2}  
\setcounter{tocdepth}{2} 
\newcommand{\bref}[1]{\mbox{(\ref{#1})}}
%-------------------------------------------------------%
\counterwithin{figure}{chapter}
\counterwithin{exemplo}{chapter}
\counterwithin{definição}{chapter}
\counterwithin{algoritmo}{chapter}


% If dedication remove the next line comment...
%\dedication{Para ...}

\college{Minha Unidade}  
\advisor{Orientador: Prof.  Fulano de Tal, UFMG \\
Supervisor: Eng. John Doe, Empresa}
\fulfilment{Relatório de Estágio Técnico do Projeto PRAE/COLTEC/UFMG} 

%\fulfilment{Monografia submetida à banca examinadora designada pelo Colegiado Didático do Curso de Graduação em Engenharia de Controle e Automação da Universidade Federal de Minas Gerais, como parte dos requisitos para aprovação na disciplina Projeto Final de Curso II} 

\submitterm{Novembro,}
\submityear{2017}

%\draft


%%%%%%%%%%%%%%%%%%%%%%%%%%%%%
% INICIO
%%%%%%%%%%%%%%%%%%%%%%%%%%%%%

\begin{document}
% Título...
\title{Título com assunto e proposta}
\author{Autor Nome Sobrenome}
\maketitle

%\input{capa}

\pagenumbering{roman} % Use Roman numbers for the frontmatter pages.

\begin{abstract}
%\input{abstract}
% !TEX encoding = UTF-8 Unicode
%\section{Abstract}

O resumo deve explicitar o assunto, a proposta e o escopo do trabalho. Ao ler o resumo o leitor deve entender de que se trata efetivamente o trabalho ou projeto, o que se aborda e como se produz evidências (simulação, experimento, dedução teórica). Normalmente a última sentença do resumo trata das evidências e discussões apresentadas ou almejadas. 


\end{abstract}

\tableofcontents
\listoffigures

\begin{acknowledgements}
%%%%%%%%%%%%%%%%%%%%%%%%%%%%%%%%%%%%%
%% Agradecimentos
%%%%%%%%%%%%%%%%%%%%%%%%%%%%%%%%%%%%%
% !TEX encoding = UTF-8 Unicode

Agradecimentos...	 % File with my agradecimentos.tex   (same folder of master)
\end{acknowledgements}

\pagenumbering{arabic} % Use arabic numbers for the mainmatter.

% Subpastas para cada capítulo
% !TEX encoding = UTF-8 Unicode
\graphicspath{{figuras/}}

\chapter{Introdução \label{cap1}}

O Relatório Técnico Final é um documento em que o leitor encontra todas as informações técnicas sobre o trabalho desenvolvido. 
O relatório deve conter os seguintes tópicos:
\begin{itemize}\item {\bf Capa ou página de título: } deve conter o nome da instituição, o título (e subtítulo) do projeto, os autores e supervisores, local e data. O título é considerado o resumo mais sintético do projeto e, portanto, é imprescindível que inclua o assunto ou o tópico e a proposta do mesmo. Verifique ainda se o título é: suficientemente preciso, fácil para ler e entender, e  estruturado para o tema e a audiência. 
\item {\bf Contra-capa: } mesma informação da capa mais as assinaturas dos autores.
\item {\bf Sumário: } Enumeração das principais divisões (capítulo, seções, artigos, etc.) do documento, na mesma ordem em que a matéria nele se sucede; visa a facilitar visão do conjunto da obra e a localização de suas partes, e, para tanto, deve aparecer no início da publicação e indicar, para cada parte, a paginação (Dicionário Aurélio, 1999).
\item {\bf Abstract:} é um resumo sucinto do trabalho que serve como um guia para a leitura do relatório. A leitura do abstract deve indicar se vale ou não a pena ler o relatório. 
O abstract pode ser de dois tipos:

\begin{itemize}
  \item Abstract descritivo que responde a questão: qual é o escopo do relatório?
    \item Abstract informativo que responde a questão: quais são os pontos mais importantes apresentados no relatório.
\end{itemize}

\item {\bf Introdução:} uma introdução bem escrita deve abordar os seguintes itens:
  \begin{itemize}
  	\item o assunto  	\item a proposta ou proposição
	\item Objetivos: 
	Enumere uma lista com bullets. Recomenda-se iniciar os itens com verbos no infinitivo. É imperativo manter o paralelismo de linguagem, i.e. se o primeiro item inicia-se com um verbo no infinitivo, todos os demais itens devem também iniciar com verbos no infinitivo!
  	\item o "background" ou fundamentos do projeto  	\item o escopo  	\item a organização do relatório 	 \item os termos chaves
  \end{itemize}
	  \item {\bf Descrição da metodologia:} descreva os métodos usados para executar o projeto.  \item {\bf Apresentação dos resultados}  \item {\bf Conclusões:} conclua baseado nos resultados apresentados, comentando todos os objetivos propostos.  \item {\bf Sugestões e recomendações}  \item {\bf Apêndices:} nomenclatura utilizada (vide exemplo) e material de suporte ou complementação do corpo do relatório, e.g. diagramas de circuitos, códigos de programas desenvolvidos, etc.  \item {\bf Referências bibliográficas}.
\end{itemize}

Um livro clássico mas difícil de ler e entender é \cite{Astrom:1970}, mas também clássico e muito bom é \cite{Astrom:1997}.

Note que figuras são referenciadas com o rótulo {\it figura} seguindo de uma referência numérica sem parênteses, e.g. Mostra-se na figura \ref{fig_cap1_MBPC_blk_diag} um diagrama em blocos de uma arquitetura de controle de processos genérica.

A referência a uma equação é feita usando numeração entre parênteses, e.g. a equação \bref{eq_piuBella} é uma das mais belas equações matemáticas.

\begin{equation}
 \frac{dy}{dt} = A y.   
 \label{eq_piuBella}
\end{equation}


\begin{figure}[!htbp]
\centering
\includegraphics[width=15cm]{cap1_MBPC_blk_diag} 
\caption{Arquitetura simplificada de um sistema de controle baseado em modelo com a função de Auditor de Processo.}
\label{fig_cap1_MBPC_blk_diag}
\end{figure}


\section{Contribuições}

As principais contribuições apresentadas neste trabalho são destacadas abaixo para cada capítulo.

\begin{itemize}

  \item \textbf{Capítulo \ref{cap1}}. Neste capítulo contribuiu-se a partir de uma revisão da literatura...
  
  \item \textbf{Capítulo \ref{cap2}}. Descreve-se...
  
  \item \textbf{Capítulo \ref{cap3}}. Apresenta-se...
  
  \item \textbf{Capítulo \ref{cap4}}. No capítulo 4 é apresentada ...
  
  \item \textbf{Capítulo \ref{cap5}}. No capítulo 5 é apresentada ...
  
\end{itemize}


Conclusões e sugestões de trabalho futuro são apresentadas no Capítulo \ref{cap6}.

% !TEX encoding = UTF-8 Unicode
\graphicspath{{figuras/}}

\chapter{Título do capitulo 2}
\label{cap2}

Texto resumo introdutório do capítulo...

\section{Introdução}


\section{Comentários finais}

% !TEX encoding = UTF-8 Unicode
\graphicspath{{figuras/}}

\chapter{Título do capitulo 3}
\label{cap3}

Texto resumo introdutório do capítulo...

\section{Introdução}


\section{Comentários finais}

% !TEX encoding = UTF-8 Unicode
\graphicspath{{figuras/}}

\chapter{Título do capitulo 4}
\label{cap4}

Texto resumo introdutório do capítulo...

\section{Introdução}


\section{Comentários finais}

% !TEX encoding = UTF-8 Unicode
\graphicspath{{figuras/}}

\chapter{Título do capitulo 5}
\label{cap5}

Texto resumo introdutório do capítulo...

\section{Introdução}


\section{Comentários finais}

% !TEX encoding = UTF-8 Unicode
\chapter{Conclusões e sugestões de trabalho futuro\label{cap6}}

\section{Conclusões}
 Os objetivos foram alcançados?
 
 
\section{Propostas de Trabalho futuro.}

Trabalho futuro.

%-------------------------------------------------------------
\appendix %-----------------------------------------------

\input{ApendiceA/ApendiceA.tex} 


%---------------------------------------------------------------------------
% Before the reference page must be inserted a \pagebreak to ensure that
% the references are place in TOC (Table of Contents) with the right page number.
% A \addcontentsline command must be used to insert the References title
% in the TOC.
%---------------------------------------------------------------------------

\addcontentsline{toc}{chapter}{Referências}

%\bibliographystyle{abnt-alf}
\bibliographystyle{alpha}
\bibliography{ref,ref_books}
%% NÃO pode haver espaço entre os databases acima!!! (:-/

\end{document}
