% !TEX encoding = UTF-8 Unicode
\graphicspath{{figuras/}}


\chapter{Metodologia para resolução de problemas}
\label{cap4}

O desenvolvimento de projeto orientado requer reuniões com orientador ou supervisor para detalhamento dos problemas e busca conjunta de soluções. Desenvolver um trabalho orientado é uma oportunidade para se aprender técnicas e procedimentos que são valiosos e agregam valor a um profissional que futuramente venha a trabalhar em projetos complexos resolvidos por equipes multidisciplinares.  Dicas são apresentadas de forma esquemática e sucinta de como abordar problemas complexos que requerem a atuação em  equipes multidisciplinares. Apresentam-se tópicos e metodologia (estruturas e procedimentos) que favorecem um  fluxo de trabalho focado em compreender, delinear e delimitar um determinado problema, para se construir propostas de soluções factíveis e acionáveis.

\section{Sete Passos para Resolver Problemas}
A metodologia para se abordar e resolver um problema é sintetizada nos seguintes  sete passos:
\begin{enumerate}[noitemsep]
	\item   Definir o problema.
	\item  Dividir em questões.
	
	\keyparbox[H]{lw=0.5}{% --- inicio
		\begin{tikzpicture}[scale=0.6,transform shape,grow=right, sibling distance=1cm, level distance=3cm]
			\node[rectangle,draw] {Problema}[edge from parent fork right,grow=right] 
			child {node[rectangle,draw] {Questão 2}}
			child {node[rectangle,draw] {Questão 1}
				child {node[rectangle,draw, text width=2.35cm] {Ameaças}}
				child {node[rectangle,draw, text width=2.35cm] {Oportunidades}}
			};
			\draw[red] (2.25,0) -- ++(1.5,-1);
			\draw[red] (2.25,-1) -- ++(1.5,1);
		\end{tikzpicture}
	}%--- fim

	\item  Eliminar todas as que não são questões chaves.
	\item Obter dados e conduzir uma análise crítica.
	\item Agrupar as constatações (fatos) e construir argumentos.
	
	\keyparbox[H]{lw=0.5}{% --- inicio
		\begin{tikzpicture}[scale=0.6,transform shape,level distance=1.5cm, level 1/.style={sibling distance=3cm}, level 2/.style={sibling distance=1cm} ]
			\node[rectangle,draw] {Tópico}[edge from parent fork down,grow=down] 
			child {node[rectangle,draw,sibling distance=1cm,align=center] {Não tem \\}
				child {node[rectangle,draw] {\phantom{mm}}}
				child {node[rectangle,draw] {\phantom{mm}}}
			}
			child {node[rectangle,draw,align=center] {Faltando\\ ainda}
				child {node[rectangle,draw] {\phantom{mm}}}
				child {node[rectangle,draw] {\phantom{mm}}}
			}
			child {node[rectangle,draw,sibling distance=1cm,align=center] {Características\\ únicas}
				child {node[rectangle,draw] {\phantom{mm}}}
				child {node[rectangle,draw] {\phantom{mm}}}
			}
			;
		\end{tikzpicture}
	}%--- fim
	\item Contar a história.
	\item Revisar e reavaliar periodicamente
\end{enumerate}

A Figura~\ref{fig:ResolucaoLogicaProb} ilustra os três aspectos de uma resolução lógica de problemas visando impacto.
\keyfigbox[H]{lw=1, c={Resolução lógica de problemas}, l=fig:ResolucaoLogicaProb}{% --- inicio
	\centering 
	\begin{tikzpicture}[grow=left, sibling distance=1.5cm, level distance=3.5cm,->,>=Straight Barb]
		\node[rectangle,draw,align=left] {Resolução lógica\\ de problemas}[edge from parent fork right,grow=right] 
		child {node[rectangle,draw,align=left, text width=2.25cm] {Induzido pelo\\ impacto}}
		child {node[rectangle,draw,align=left, text width=2.25cm] {Focalizado\\\phantom{mmm} }}
		child {node[rectangle,draw,align=left, text width=2.25cm] {Baseado \\ em fatos}};
		\node at (5.65,0) [dart,dart tip angle=120,draw,red!50!gray]{\phantom{|}};
		\node at (6.75,0)   [draw,starburst, fill=pink,rotate=90]{Impacto}; 
	\end{tikzpicture}
}%--- fim


\section{Questões características de uma boa resolução de problemas}

A metodologia (modelos e procedimentos)  de resolução de problemas se inicia com a definição e delimitação do que precisa ser resolvido.  Uma forma pragmática para construir  essa definição é elaborando questões que devem ser:

\begin{enumerate}
	\item  Provocativas de pensamentos, não um fato.
	\item  Específicas e não gerais.
	\item  Debatíveis.
	\item  Acionáveis (desencadeadoras de ações).
	\item  Focalizadas.
\end{enumerate}

O processo de resolução de um problema deve ser \textbf{MECE} --- \textsc{Mutuamente
Exclusivo, Coletivamente Exaustivo}. As questões precisam ser pensadas e identificadas o mais  isoladamente possível e de forma exaustiva, i.e. cobrindo todos os aspectos imagináveis. Não se preocupar, neste momento, se as questões tem ou não impacto, são ou não viáveis. Depois de relacionadas, elimina-se aquelas  questões consideradas irrelevantes, mas sabendo que foram cogitadas. 

\begin{enumerate}
	\item  Para "quebrar" o problema \faBrain
	\begin{enumerate}
		\item   trabalho pode ser dividido e gerenciado em pedaços \faShapes\; \faShare*.
		\item    prioridades podem ser definidas \faListOl.
		\item    responsabilidades podem ser alocadas \faHandshake\;\faAddressCard.
	\end{enumerate}
	
	\item   Para garantir que a integridade da solução do problema seja mantida \faFileContract.
	
\end{enumerate}

\section{Métodos de divisão do problema}

\begin{itemize}
	\item  \textbf{Árvore de questões} \faSitemap: quando há pouco conhecimento prévio \tikz[baseline={([yshift=-.8ex]current bounding box.center)}] \node{\faEyeSlash};
	\item  \textbf{Induzido por hipóteses} $A\Rightarrow B$: quando há experiência prévia \tikz[baseline={([yshift=-.8ex]current bounding box.center)}] \node{\faEye};-- \texttt{Se A, então B}.
\end{itemize}


\section{Elementos chaves para a resolução de problemas em grupo}

\begin{enumerate}
	\item  Organize a liderança  \faUsersCog \; $\rightleftarrows$\; \faUserTie;
	\item  Divida o trabalho (sub-grupos) \faSitemap\; $\rightarrowtail$\;\faUsers ;
	\item  Ouça \faComments, reflita \;\faLightbulb, tome nota \faEdit ;
	\item  Sintetize (secretário(a)) \faTasks;
	\item  Reveja \faUndo\;  e pense a respeito \faDiagnoses \; \faCommentsDollar.
\end{enumerate}

\section{Organização e Estrutura}

\begin{itemize}
	\item  Agrupe as constatações fáceis e determine como elas se relacionam com as questões chaves \faObjectUngroup[regular]\;  $\Leftrightarrow$\; \faKey;
	\item  Aborde cada questão numa estrutura \emph{top} -- \emph{down} (geral $\to$ detalhe) \faSitemap;
	\item  Mantenha a mensagem focalizada \faMicroscope;
	\item  Seja \textbf{MECE}\footnote{MECE: Mutuamente
			Exclusivo, Coletivamente Exaustivo.} \faBuromobelexperte.
\end{itemize}

\section{Esquemas de organização e apresentação dos tópicos}

Problemas são organizados de duas formas: 
\begin{enumerate}
	\item Agrupamento \faObjectGroup[regular]
	\keyfigbox[H]{lw=1, c={Estrutura de agrupamento}, l=EstruturaAgrupamento}{% --- inicio
		\centering
		\begin{tikzpicture}[scale=1,transform shape,level distance=1.5cm, level 1/.style={sibling distance=3cm}, level 2/.style={sibling distance=1cm} ]
			\node[rectangle,draw] {Tópico}[edge from parent fork down,grow=down] 
			child {node[rectangle,draw,sibling distance=1cm,align=center] {Não tem \\}
				child {node[rectangle,draw] {\phantom{mm}}}
				child {node[rectangle,draw] {\phantom{mm}}}
			}
			child {node[rectangle,draw,align=center] {Faltando\\ ainda}
				child {node[rectangle,draw] {\phantom{mm}}}
				child {node[rectangle,draw] {\phantom{mm}}}
			}
			child {node[rectangle,draw,sibling distance=1cm,align=center] {Características\\ únicas}
				child {node[rectangle,draw] {\phantom{mm}}}
				child {node[rectangle,draw] {\phantom{mm}}}
			}
			;
		\end{tikzpicture}
	}%--- fim
	\item Argumentação lógica \tikz[baseline={([yshift=-.8ex]current bounding box.center)}] \node{\faEye}; -- \texttt{Se A, então B}.
	
	\keyfigbox[H]{lw=1, c={Estrutura de argumentação lógica com encadeamento de causalidade.}, l=EstruturaArgumentacao}{% --- inicio
		\centering
		\begin{circuitikz}[ >=triangle 45]
			\tikzset{%
				block/.style    = {draw, thick, rectangle, minimum height = 2em,
					minimum width = 3.5em, node distance = 4cm},
				sum/.style      = {draw, circle, node distance = 2cm}, % Adder
				input/.style    = {coordinate}, % Input
				output/.style   = {coordinate} % Output
			}
			\draw
			% Drawing the blocks
			node [block ] (G1) {Declaração}
			node [block, right of=G1] (G2) {Comentário}
			node [block, right of=G2] (G3) {Conclusão}
			;
			% Conectando os  blocos.
			\draw[->](G1) -- (G2);
			\draw[->](G2) -- (G3);
		\end{circuitikz}
	}%--- Fim
	
\end{enumerate}

A Figura~\ref{fig:ContandoHistoriaRP} ilustra duas formas, usando tabelas,  de como apresentar a história do problema e as propostas de soluções. 

\begin{keysubfigs}{1}{c={Contando a história da resolução do problema.},l=fig:ContandoHistoriaRP}
	\keyfigbox{lw=1,c={Encadeando a situação atual, complicações e resolução.},	l=fig:SubfigA}{% --- FigBox A
		\centering
		\begin{tblr}{ 
				hlines={gray!50},
				vlines={gray!50},
				row{1}= {bg=azure3, fg=white, font=\sffamily},
				colspec={Q[m,1] Q[m,1] Q[m,1]},
			}
			\textbf{Situação\hfill \faArrowCircleRight} & \textbf{Complicação\hfill \faArrowCircleRight} & \textbf{Resolução}\\
			{O estado atual dos negócios\\ $\bullet$ O que?}
			& {O problema que está sendo resolvido\\ $\bullet$ Por que?}
			& {A Solução\\ $\bullet$ Solução a curto prazo\\ $\bullet$ Solução a longo prazo }
		\end{tblr}
	}
	\keyfigbox{lw=1,c={Conceito CCCA: Objeto alvo $\longmapsto$ Audiência},l=fig:CCCAalvoAudiencia}{% --- FigBox B
				\centering
		\begin{tblr}{ 
				hlines={gray!50},
				vlines={gray!50},
				width=1.3\textwidth,
				row{1}= {bg=azure3, fg=white, font=\sffamily},
				colspec={Q[m] Q[m] Q[m] Q[m]},
			}
			\textbf{Consciência\quad \faArrowCircleRight}& \textbf{Convicção\quad \faArrowCircleRight}& \textbf{Coragem\quad   \faArrowCircleRight}& \textbf{Ação}\\
			{Descreva a situação} & {Forneça a solução}& {Crie visões} & {Mostre o caminho}
		\end{tblr}
		
	}
\end{keysubfigs}


\section{Análise de Impacto}

A Figura~\ref{fig:AnaliseImpacto} ilustra uma representação gráfica de tópicos ou questões a serem resolvidos e uma análise comparativa de custo ou dificuldade de implementação versus impacto ou relevância da solução.  As questões acima da diagonal tracejada apresentam maior impacto   em relação ao custo ou dificuldade e, portanto, devem ser priorizadas. 

\keyfigbox[H]{lw=1,c={Esquema gráfico para análise de impacto.},l=fig:AnaliseImpacto}{%
	\centering
	\begin{tikzpicture}
		% coordinate system
		\draw[-latex] (-0.5,0) -- +(5, 0) node [below,pos = 0.7, align = center, font=\footnotesize] {Custo \$ ou  dificuldade};
		\draw[-latex] (0,-1) -- +(0, 5) node [left=1em,pos=0.75, align = center, rotate=90, font=\footnotesize] {Impacto};
		% --- bissetriz
		\draw[gray, densely dashed] (0,0) -- (4,4);
		% --- tópicos
		\node at (7.5,2) [ rectangle split,rectangle split part align=left, rectangle split parts=8,font=\footnotesize]{%
			\textbf{Tópicos}
			\nodepart{two}  \color{verde} \encircled{1} Fácil com alto impacto
			\nodepart{three} \color{azul}\encircled{2} Fácil com baixo impacto
			\nodepart{four}  \color{azcl}\encircled{3} Custoso com alto impacto
			\nodepart{five}  \color{laranja}\encircled{4} Custo e impacto moderados
			\nodepart{six}   \color{pink!50!gray}\encircled{5} Fácil com baixo impacto
			\nodepart{seven} \color{violet}\encircled{6} Difícil com baixo impacto
			\nodepart{eight} \color{red}\encircled{7} Impacto negativo, e.g. plágio
		};
		% --- posicionado os tópicos 
		\node at (0.5,3) [draw, circle,fill=verde!50, inner sep=2pt]{1};
		\node at (0.5,1) [draw, circle,fill=azul!50, inner sep=2pt]{2};
		\node at (3,3.5) [draw, circle,fill=azcl!50, inner sep=2pt]{3};
		\node at (1.75,1.75) [draw, circle,fill=laranja!50, inner sep=2pt]{4};
		\node at (1,0.5) [draw, circle,fill=pink!50, inner sep=2pt]{5};
		\node at (3.25,0.5) [draw, circle,fill=violet!50, inner sep=2pt]{6};
		\node at (0.3,-0.5) [draw, circle,fill=red!50, inner sep=2pt]{7};
	\end{tikzpicture}
}% --- Fim FigBox


\section{Preparando-se para uma entrevista ou reunião \faPeopleArrows}
Estar sempre preparado para responder uma questão em 30 segundos sobre o projeto ou problema em que se está trabalhando é uma recomendação valiosa. A dica é sempre pensar como se estivesse elaborando um título, i.e. o resumo mais sintético daquilo que estamos trabalhando deve conter o assunto e uma proposta de resolução. Se este resumo extremamente sintético for atraente e sagaz seja um cliente ou um orientador demonstrará o devido interesse e o tempo para conhecer mais detalhes. 

\begin{itemize}
	\item  Instruções
	\begin{itemize}
		\item    Defina os tópicos para falar.
		\item    Apoie com pelo menos 2 níveis de uma pirâmide .
		\item    Salve os resultados (sumário).
		\item    Esteja preparado para dizer alguma coisa em um minuto.
		\item    Anote qualquer dificuldade para discutir.
	\end{itemize}
	\item  Sugestões
	\begin{itemize}
		\item    Defina uma meta realista para uma configuração.
		\item    Defina um pensamento direcionador, que  é uma questão chave estabelecida na definição do problema.
		\item    Considere audiência (Qual questão deve ser colocada?).
		\item    Use agrupamentos;
	\end{itemize}
\end{itemize}

\section{Entrevista}

Entrevistas são oportunidades não apenas para se obter informações para resolver problemas, mas também oportunidade para se construir um relacionamento vantajoso de cooperação.
Portanto, é importante planejar uma entrevista  visando resultados com uma sequência lógica de construção do entendimento global, e.g.:


\emph{Questões $\to$ Hipóteses $\to$ Disponibilidade de Dados (novos dados de
	entrevistas) $\to$  Análise $\to$ Redação de relatórios.}

\paragraph{Diálogo Guiado}

A reunião com um cliente ou orientador deve ser pensada como uma oportunidade para se obter informação para resolver determinado problema, além de se construir um relacionamento amigável que permita uma antecipação vantajosa de informações durante uma busca de soluções.

Preparando-se para uma entrevista, organize:
\begin{itemize}
	\item Sequência de tópicos
	\begin{itemize}
		\item Importância
		\item Sensitividade
	\end{itemize}
	\item Antecipação
	de entraves e soluções ponderáveis.
	\item Documentação
	existente sempre ao alcance.
	\item Necessidades
	já identificadas e justificadas.
\end{itemize}


\paragraph{Guia de entrevista} Pensar sobre estratégias que auxiliem a solução do problema e abordagens que considerem os diversos aspectos que contextualizam o problema, e.g.: 

\begin{enumerate}
	\item Visão de mercado
	\begin{enumerate}
		\item Estrutura da indústria
		\item Competitividade
		\item Desempenho financeiro
	\end{enumerate}
	\item Análise de ambientes
	\begin{enumerate}
		\item Ambiente externo
		\subitem Oportunidades
		\subitem Ameaças
		\item Ambiente interno
		\subitem Forças
		\subitem Fraquezas
	\end{enumerate}
	\item Oportunidades
	\begin{enumerate}
		\item Fator chave de compra
		\item Alavancas de lucratividade
		\item Custo
	\end{enumerate}
	\item Competidores
	\begin{enumerate}
		\item Prioridades
		\item Relacionamento (fraco/forte)
		\item Desenvolvimento antecipado
		\item Não-tradicional
	\end{enumerate}
\end{enumerate}


%%%%%
\section{Para se ter em mente}

\begin{itemize}
	\item  Ouça o problema -- esteja seguro de que a questão certa está sendo
	respondida.
	\item Coloque uma estrutura na sua frente -- pense em uma estrutura
	possuindo 4-5 questões chaves que você precisa responder antes de
	sintetizar a resposta sobre a questão total.
	\item  Proceda em um estilo organizado -- conclua uma questão da estrutura
	e chegue em um ponto de vista a este respeito antes de partir para o
	próximo.
	\item  Volte um passo atrás periodicamente -- sumarize o que tem sido
	aprendido e o que as implicações parecem ser.
	\item  Comunique a cadeia lógica do seu pensamento -- as alternativas
	consideradas e as rejeitadas devem ser relatadas.
	\item  Peça informações com ``bom senso''\footnote{\label{rodape:bom senso}Bom senso é considerado o dom mais bem distribuído por Deus a humanidade, pois todos \emph{acham} que receberam muito. Considerando uma  inexorável cegueira psicológica inerente,  perdemos a noção da diminuição de nosso nível de bom senso durante debates acalorados ou dificuldades de foco no que importa, arriscando desnecessariamente com demandas descabidas!} -- pense porque a informação é
	necessária e os caminhos para obtê-las, de preferência, sem custo ou a
	custo baixo.
	\item  Mantenha a cabeça aberta.
	\item  Demonstre claramente o pensamento lógico do
	processo ao invés de simplesmente apresentar uma solução.
	\item  Use o senso comum, mas com cautela, pois senso comum tende a ser confundido com \emph{bom senso}${}^{\ref{rodape:bom senso}}$.
	\item  Relaxe e aprecie o processo (:-).
\end{itemize}

\section{Enganos comuns}

\begin{itemize}
	\item Falhar em compreender onde termina uma questão, respondendo a questão
	errada.
	\item  Proceder de maneira arriscada, i.e., não identificar as questões
	principais que devem ser examinadas ou ficar ``pulando'' de uma
	questão para outra.
	\item Levantar um grande número de questões sem explicar porque esta
	informação é necessária.
	\item  Forçar-ajustar poucas ferramentas familiares para toda questão em
	análise.
	\item  Ser incapaz de sintetizar um ponto de vista baseado na informação
	fornecida.
\end{itemize}

\section{Comentários finais}

Listas com sugestões  para abordar e resolver problemas de consultoria foram apresentados. A resolução de problemas seja em equipes multidisciplinares ou com um orientador e supervisor é um processo lógico que requer foco e preparação. As dicas fornecidas esquematicamente servem como um lista de verificação para auxiliar e favorecer um desenvolvimento harmonioso de atividades que culminam em um relatório técnico adequadamente redigido.